\PassOptionsToPackage{unicode=true}{hyperref} % options for packages loaded elsewhere
\PassOptionsToPackage{hyphens}{url}
%
\documentclass[12pt]{article}
\usepackage{lmodern}
\usepackage{amssymb,amsmath}
\usepackage{ifxetex,ifluatex}
\usepackage{fixltx2e} % provides \textsubscript
\ifnum 0\ifxetex 1\fi\ifluatex 1\fi=0 % if pdftex
  \usepackage[T1]{fontenc}
  \usepackage[utf8]{inputenc}
  \usepackage{textcomp} % provides euro and other symbols
\else % if luatex or xelatex
  \usepackage{unicode-math}
  \defaultfontfeatures{Ligatures=TeX,Scale=MatchLowercase}
\fi
% use upquote if available, for straight quotes in verbatim environments
\IfFileExists{upquote.sty}{\usepackage{upquote}}{}
% use microtype if available
\IfFileExists{microtype.sty}{%
\usepackage[]{microtype}
\UseMicrotypeSet[protrusion]{basicmath} % disable protrusion for tt fonts
}{}
\IfFileExists{parskip.sty}{%
\usepackage{parskip}
}{% else
\setlength{\parindent}{0pt}
\setlength{\parskip}{6pt plus 2pt minus 1pt}
}
\usepackage{hyperref}
\hypersetup{
            pdftitle={The recruitment dynamics of the nylon shrimp Heterocarpus reedi, effects of climate and predation off Chile},
            pdfborder={0 0 0},
            breaklinks=true}
\urlstyle{same}  % don't use monospace font for urls
\usepackage[left=2.5cm,right=2.5cm,top=2.5cm,bottom=2.5cm,headheight=12pt,letterpaper]{geometry}
\usepackage{graphicx,grffile}
\makeatletter
\def\maxwidth{\ifdim\Gin@nat@width>\linewidth\linewidth\else\Gin@nat@width\fi}
\def\maxheight{\ifdim\Gin@nat@height>\textheight\textheight\else\Gin@nat@height\fi}
\makeatother
% Scale images if necessary, so that they will not overflow the page
% margins by default, and it is still possible to overwrite the defaults
% using explicit options in \includegraphics[width, height, ...]{}
\setkeys{Gin}{width=\maxwidth,height=\maxheight,keepaspectratio}
\setlength{\emergencystretch}{3em}  % prevent overfull lines
\providecommand{\tightlist}{%
  \setlength{\itemsep}{0pt}\setlength{\parskip}{0pt}}
\setcounter{secnumdepth}{0}
% Redefines (sub)paragraphs to behave more like sections
\ifx\paragraph\undefined\else
\let\oldparagraph\paragraph
\renewcommand{\paragraph}[1]{\oldparagraph{#1}\mbox{}}
\fi
\ifx\subparagraph\undefined\else
\let\oldsubparagraph\subparagraph
\renewcommand{\subparagraph}[1]{\oldsubparagraph{#1}\mbox{}}
\fi

% set default figure placement to htbp
\makeatletter
\def\fps@figure{htbp}
\makeatother

\usepackage{booktabs}
\usepackage{longtable}
\usepackage{array}
\usepackage{multirow}
\usepackage{wrapfig}
\usepackage{float}
\usepackage{colortbl}
\usepackage{pdflscape}
\usepackage{tabu}
\usepackage{threeparttable}
\usepackage{threeparttablex}
\usepackage[normalem]{ulem}
\usepackage{makecell}
\usepackage{lineno}
\usepackage{placeins}
\usepackage{authblk}

\linenumbers
\linespread{1}
\author[1]{Camila Sagua}
\author[1,2,*]{Luis A. Cubillos}
\author[3]{Cristian M. Canales}
\author[4]{Ruben Alarcón-Muñoz}
\affil[1]{Magister en Ciencias con mención Pesquerías, Facultad de Ciencias Naturales y Oceanográficas, Universidad de Concepción, Concepción, Chile.}
\affil[2]{Centro de Investigación Oceanográfica COPAS Sur-Austral, Lab. EPOMAR y Departamento de Oceanografía, Universidad de Concepción, Casilla 160-C,Concepción, Chile.}
\affil[3]{Escuela de Ciencias del Mar, Pontificia Universidad Católica de Valparaíso, Valparaíso, Chile.}
\affil[4]{Doctorado en Ciencias con mención en Manejo de Recursos Acuaticos Renovables, Universidad de Concepción, Concepción, Chile.}
\affil[*]{Corresponding author. Email: lucubillos@udec.cl}
\linespread{1.5}
\usepackage{lineno} \usepackage{placeins} \linenumbers \linespread{1.5}
\usepackage{booktabs}
\usepackage{longtable}
\usepackage{array}
\usepackage{multirow}
\usepackage{wrapfig}
\usepackage{float}
\usepackage{colortbl}
\usepackage{pdflscape}
\usepackage{tabu}
\usepackage{threeparttable}
\usepackage{threeparttablex}
\usepackage[normalem]{ulem}
\usepackage{makecell}
\usepackage{xcolor}

\title{The recruitment dynamics of the nylon shrimp \emph{Heterocarpus reedi},
effects of climate and predation off Chile}
\author{}
\date{\vspace{-2.5em}}

\begin{document}
\maketitle

\hypertarget{abstract}{%
\section{Abstract}\label{abstract}}

Climate variability and predation influence the fluctuations in the
recruitment of exploited marine populations. This study analyses the
dynamics of the recruitment of the nylon shrimp (\emph{Heterocarpus
reedi} Bahamonde 1955) over the period 1968 and 2015, considering the
influence of climate variability and the biomass of common hake
(\emph{Merluccius gayi}), as a proxy for predation in its distribution
area. We collected the Humboldt Current Index (HCI) and Southern
Oscillation Index (SOI) time series as climate variables and estimates
of recruitment and spawning biomass of nylon shrimp, as well as biomass
of Chilean hake. Annual deviations in nylon shrimp recruitment showed
increased sensitivity to climate variability from the late 1990s
onwards, expressed through a significant cumulative correlation over
time (\(P<0.05\)). Generalized Linear Models showed that climatic
variables and biomass of common hake were responsible for the
recruitment rate (\(r^2=0.873\), \(P<0.05\)). A path diagram with
structural equation models (SEM) showed that the recruitment rate is
being influenced by the greater or lesser biomass of common hake, either
through the consumption of juveniles in the year of formation of the
annual class or on adult spawners, in interaction with climate
variability.

\textbf{Key words}: Recruitment, shrimp, climate, predation,
sensitivity, trawl fishery

\hypertarget{introduction}{%
\section{Introduction}\label{introduction}}

In general, fishing and/or the environment influence the large
fluctuations experienced by the stocks, directly or indirectly affecting
the magnitude of recruitment, which expresses through the entry of
juveniles into the exploitable fraction (Gómez et al., 2012; Hsieh et
al., 2006). Likewise, the pre-recruitment phases can be sensitive to
favorable or unfavorable environmental factors (Hidalgo et al., 2011;
Perry et al., 2010; Planque et al., 2010). Fluctuations in recruitment
are caused not only by climate forcing and fishing but also due to
ecological interactions such as predator-prey ({\textbf{???}};
{\textbf{???}}; {\textbf{???}}; {\textbf{???}}), particularly the
shrimp-gadoid relationship (Björnsson et al., 2017, 2011; Drinkwater et
al., 2010; Worm and Myers, 2003).

Climate variability occurs on different time scales, from the seasonal,
interannual cycle with irregular periods of 1 to 3 years associated with
El Niño-Southern Oscillation (ENSO), aperiodic decadal variability
between 5 to 50 years to centennial and longer-term periods (Chavez et
al., 2003; Overland et al., 2010; Perry et al., 2010). There is evidence
that this variability influences the status and functioning of marine
ecosystems and is closely related to population distribution, migration,
and abundance (Lehodey et al., 2006). It has also been shown that
fishing and other human activities have an impact on exploited and
unexploited species (Hsieh et al., 2006; Planque et al., 2010), and can
modify the structure, size, and functioning of ecosystems (Cury et al.,
2000).

In the upwelling ecosystem of the Humboldt Current System, there have
been seasonal and decadal changes in populations of small pelagic fish
such as anchovy (\emph{Engraulis ringens}) and sardine (\emph{Sardinops
sagax}), which demonstrate the almost immediate impact of the
environment and synchrony over a wide latitudinal range (Alheit and
Niquen, 2004; Cubillos et al., 2007; Yañez et al., 2008). In the
central-southern zone, the inter-annual variability of ENSO influences
pelagic fish populations, for example, on the recruitment of the common
sardine \emph{Strangomera bentincki} (Gómez et al., 2012). El Niño and
La Niña considerably modify the neritic environment, and we observed
that in cold years (La Niña) there is an increase in the recruitment of
typical sardines as a result of an increase in biological production
(Parada, 2013). In turn, climatic conditions influence faunal changes in
plankton, zooplankton, and predators, mainly in the pelagic ecosystem
(Alheit and Niquen, 2004; Ayón et al., 2011; Chavez et al., 2003; van
der Lingen et al., 2009; Xu et al., 2019), whose greater or lesser
predation would have a direct effect on the benthic and demersal system
(Cury et al., 2000). However, few studies have aimed at assessing the
role of climate influence on benthic crustaceans distributed in deep
waters over the continental platform and upper slope.

Nylon shrimp inhabit the continental platform and slope along the coast
of Chile, between 200 and 500 m deep, associated with the Peru-Chile
Subsurface Current flowing towards the pole, and the mixing of water
masses between the Antarctic Intermediate Water (oxygenated and cold
water, 11-12 ºC) and the Equatorial Subsurface Water (O\(_2\)
\textless{}1 ml L\(^{-1}\), 12-13 ºC, 35 psu) (Arana, 2012; Bahamonde
and Henrı'quez, 1970; Silva, 2012). South of its distribution, river
discharges, and upwelling events determine the availability of organic
matter that could favor adult feeding. Likewise, the northward
circulation of the Humboldt Current may explain the presence of
juveniles in the north of its current distribution through processes of
survival and recruitment by larval or juvenile dispersal (C Canales et
al., 2016). C Canales et al. (2016) suggested that the shrimp population
is likely to be a metapopulation structure with at least two subunits
located north and south of 32°S, and whose connectivity would be
explained by larval drift.

The shrimp fishery is located between 25ºS and 37º10'S, with two stock
units (C. M. Canales et al., 2016; Montenegro and Branco, 2016): a
northern stock between 25ºS and 32ºS and a southern stock from 32ºS to
37º10'S (Fig. 1). The fishery involves a fleet of small industrial
trawlers on the slope and the continental platform. The available data
have made it possible to evaluate this resource from 1940 to 2015
through stock assessment models structured by length (C.M. Canales et
al., 2016). Biomass indices are available from biomass evaluation by the
swept area method, which show an increasing trend south of 32°S from
2006 (Acuña et al., 2012). According to (C Canales et al., 2016),
positive trends in biomass could be influenced by the reduction in
exploitation levels, the closure of some fishing areas since 2000, and
by the depletion of Chilean hake (\emph{Merluccius gayi gayi}) after
2003. The common hake is one of the most abundant and important
predators of the demersal system in which shrimp and prawns live (Arana
and Williams, 1970; Arancibia and Neira, 2008; Cubillos, 2007), and it
could be an additional factor influencing the variability of long-term
recruitment of nylon shrimp.

According to the above, understanding and taking into account these
sources of variability and their interactions represent a challenge for
modern fisheries management. This work aimed to evaluate the hypothesis
that the changes observed in the recruitment and biomass of nylon shrimp
could be influenced by their population dynamics and the trophic
interaction with common hake in a changing environment. \FloatBarrier

\hypertarget{material-and-methods}{%
\section{Material and Methods}\label{material-and-methods}}

\hypertarget{source-of-data-and-information}{%
\subsection{Source of data and
information}\label{source-of-data-and-information}}

We obtained the time series of spawning biomass and shrimp recruitment
from (C.M. Canales et al., 2016), who considered data from the northern
and southern fishery unit in a spatial stock assessment model. For the
analysis, we discarded the estimates of spawning biomass and recruitment
for the years 1945 to 1960 because these early estimates do not account
for variability but rather for expected recruitment. Besides, one of the
climate indices began in 1968, conditioning the study period (see
below).

The variations in the recruitment of the northern and southern fishing
units are identical, with slight differences in the time series. This is
due to the stock assessment model that considers shrimp recruitment to
be a function of the same signal and spatially segregated by a
proportion that defines connectivity between areas (C.M. Canales et al.,
2016). Using in this context, the sum of northern and southern
recruitment in subsequent analyses.

We used two climate indices to study the influence of environmental
variability on shrimp recruitment, the Humboldt Current Index (HCI)
(Blanco-Garcı'a, 2004), and the Southern Oscillation Index (SOI). The
HCI measures the atmospheric circulation between Rapa Nui (27º6'16.8'`S,
109º21'37.7'`S) and Antofagasta (23º38'39'`S, 70º24'39''S) and it is an
index of the decadal climate variability of the Humboldt Current System
\href{http://www.bluewater.cl/HCI/hci.html}{}. The SOI is a standardized
index based on observed sea level pressure differences between Tahiti
and Darwin, Australia
\href{https://www.ncdc.noaa.gov/teleconnections/enso/indicators/soi/}{},
which measures Walker's atmospheric circulation. These indicators make
it possible to detect decadal and interannual changes in the climate
that could restructure the ecosystem, which is associated with
long-lasting periods of hot or cold temperature anomalies, related to
the approach or retreat of warm subtropical oceanic waters to the coast
of Chile (Alheit and Niquen, 2004).

\hypertarget{hake-biomass-as-a-proxy-for-predation-on-shrimp}{%
\subsection{Hake biomass as a proxy for predation on
shrimp}\label{hake-biomass-as-a-proxy-for-predation-on-shrimp}}

We used hake biomass as a proxy to study the effects of predation on
nylon shrimp. To analyze the fraction of the biomass that influences the
predation of nylon shrimp, we considered the biomass estimates of
Chilean hake by age reported by Tascheri et al. (2017) and covering the
period 1968-2015. We obtained the age selection of Chilean hake preying
on shrimp from the relationship between predator weight and prey weight
using the Ursin size-selection index (Ursin, 1973). For this purpose, we
used data on stomach contents of common hake documented by Arancibia et
al. (1998), where we calculated prey size-selectivity using the
following expression:

\[\alpha(jx,is) = \exp \left (   -\frac{ ( \log(W_{x,j}/W_{s,i})-\eta )^2}{2\sigma^2} \right )\]

where \(W_{x,j}\) is the average weight of predator \(x\) at age \(j\),
\(W_{s,i}\) is the average weight of prey \(s\) at age \(i\). The
constant \(\eta\) represents the average weight ratio between prey and
predator, and \(\sigma\) represents the range of prey size in the
predator's diet. Once obtained this index, we analyzed the age range of
the Chilean hake affecting the size-selectivity of shrimp.

\hypertarget{recruitment-sensitivity}{%
\subsection{Recruitment sensitivity}\label{recruitment-sensitivity}}

We removed the mean from each time series of the variables used and
divided it by the standard deviation. We used these anomalies to detect
the sensitivity of shrimp recruitment to climatic indices HCI and SOI,
and to hake biomass (H) at the time of formation of the annual classes
(two years earlier). A proxy for increased sensitivity overtime was to
calculate cumulative correlations over time. The cumulative correlation
analysis began with the first 5 years of the series (1968-1972) and then
added one year sequentially until all available years (1968-2015, n=48)
were completed (Cahuin et al., 2013). The critical correlation
coefficients for the cumulative correlation decrease with the addition
of each year, which allowed to know the significance of the cumulative
correlation in time, starting from \(r = ±0.878\) (\(P = 0.025\),
\(df = 3\) years) to \(r = ±0.288\) (\(P = 0.025\), \(df = 46\) years).

\hypertarget{modeling-the-shrimp-recruitment-rate}{%
\subsection{Modeling the shrimp recruitment
rate}\label{modeling-the-shrimp-recruitment-rate}}

We used Generalized Linear Model to model the shrimp recruitment rate as
a function of the spawning stock, climate variables, and common hake
biomass. We used the log-recruitment rate \(log(R_t/S_{t-ar})\) as a
dependent variable, where \(R_t\) is the recruitment at year \(t\),
\(S_{t-ar}\) is the spawning biomass at year \(t-ar\), and \(ar\) is the
age of recruitment (\(ar = 2\)). Spawning biomass was the sum of the
north and south zones, i.e., \(St = S_{t,north} + S_{t,south}\) (C.M.
Canales et al., 2016). We analyzed the phase or lagging effects of
climate variables and hake biomass. We consider a phasing effect when a
variable acts in the year of recruitment and a lagging effect when a
variable acts two years before recruitment, i.e., in the year of
formation of the annual class. We utilized GLM with link identity
(McCullagh and Nelder, 1989) and the package MASS of Venables and Ripley
(2002) and considered nine models differentiated in the combination of
predictor variables. The best model was selected by applying the
Akaike's information criterion (AIC) (Akaike, 1973) and weighed
according to Buckland et al. (1997).

\hypertarget{path-diagram}{%
\subsection{Path diagram}\label{path-diagram}}

Structural equation modeling (SEM) allowed us to understand the phase or
lagging relationship between nylon shrimp recruitment, climate indices,
and hake biomass as a proxy for predation. This technique (SEM) is a
statistical approach for parameterizing and testing causal models that
describe hypothetical relationships between multiple variables, to solve
a set of equations involved in a path diagram (Grace et al., 2010; Grace
and Bollen, 2005). Because SEM, based on maximum likelihood, uses a
correlation matrix, it assumes that all relationships are linear and
additive (Grace and Bollen, 2005). As in the GAM models, the recruitment
rate could be related to the effects of climate variability and
predation in the phase (lag = 0, or direct effects), or in lag (lag = 2)
during the year of formation of the annual classes.

\FloatBarrier

\hypertarget{results}{%
\section{Results}\label{results}}

During the study period, the Humboldt Current Index (HCI) showed
positive values in the periods 1970-1978 and 1997-2015, associated with
a strengthening of the Pacific Anticyclone (Fig. 2). Between 1979 and
1996, the index remained negative, implying a period of warmer
conditions. In turn, the Southern Oscillation Index (SOI) shows high
inter-annual variability with El Niño events expressed with negative
values of the SOI in the years 1972, 1977-1980, 1982-84, 1986-87,
1990-95, 1997-98, 2002-06, 2009, 2012, 2014 and 2015 (Fig. 2). It also
identifies La Niña events (positive SOI values) in the years 1970,
1973-76, 1981, 1985, 1988-89, 1996, 1999-2001, 2007-08, 2010-11 and
2013. It should be noted that the length of a period in which negative
values prevail between 1977 and 1997 is consistent with the same period
identified in the HCI (Fig. 2).

Anomalies of recruitment for both stocks were similar to each other,
showing negative fluctuations for the period from 1976 to 1998, and then
remaining at positive values (Fig. 3A). However, we observe fluctuations
in the spawning biomass of the northern stock that differs from that of
the southern stock, the latter being more fluctuating throughout the
period where negative values prevail for years similar to those observed
in the recruitments (1976-1998) (Fig. 3B).

The Chilean hake biomass has a sustained increase from 1970 to mid-90s,
dramatically decreasing from 2003 to 2004 and remain at low values from
2005 to 2015. Thus, we observe opposite trends between the spawning
biomass of shrimp and the biomass of common hake (Fig. 3C). Along with
the above, from the model of predation hake-shrimp, the fraction of
shrimp that goes from 0 to 3 years is depredated by all ages of common
hake, however, from age 2 + we observe a higher selectivity by hake over
4 years (Table 1).

The recruitment was not sensitive to the variability represented by SOI
throughout the study period (Fig. 5). However, the sensitivity began to
be significant and positive with the HCI, mainly between 1981-1987 and
1994-2013. The sensitivity of shrimp recruitment to changes in the
biomass of common hake showed a significant negative correlation in the
periods 1974-1975, 1978-1988, 1991-1994, and from 2005 to 2013.

Out of nine GLM models, the models that considered shrimp spawning
biomass and hake biomass in the year of recruitment, along with the
climate indicator (HCI or SOI) in the year of annual class formation,
better explained the variations in shrimp nylon recruitment during the
study period (Table 2). The models M4, M5, and M6, which include the
Chilean hake effects on the year of recruitment plus either delayed or
direct effects of the HCI, were the models that best represented the
variation in shrimp recruitment. However, of the models selected, the
model with the lowest AIC was the M6 model, which considers the negative
effects of the Chilean hake biomass directly on the recruitment rate,
and to a lesser extent the indirect effects of the HCI on the year of
recruitment (Table 2).

The SEM analysis was based on the linear correlation between variables
(Table 3) and showed that Chilean hake biomass (Hake) had significant
negative effects on shrimp spawning biomass in the year of recruitment
and annual-class formation (Hake.lag). To a lesser extent, we observed a
lagged effect of the HCI on the spawning biomass (Figure 6). There is
evidence that the predator (Hake) was negatively and significantly
related to the spawning biomass in the year of recruitment, and in the
shrimp recruitment rate, in turn, Hake.lag and HCI.lag related to the
shrimp spawning biomass with a two-year lag, the first negatively and
significantly, and the second positively, but with low significance.

\FloatBarrier

\hypertarget{discussion}{%
\section{Discussion}\label{discussion}}

The challenge for modern fisheries management is to understand and take
into account environmental variables and their interactions with
exploited populations, leading to a more holistic view of the processes
that influence the dynamics of exploited marine populations. In this
study, we found that the recruitment of nylon shrimp is being influenced
by the higher or lower biomass of common hake, either through
consumption of juveniles in the year of formation of the annual class or
on adult spawners, in interaction with the interannual climate
variability reflected through the ENSO-associated HCI during spawning.

The low influence of the spawning biomass itself on the recruitment rate
and the climatic indicators HCI and SOI on shrimp spawning biomass in
the year of formation of the year class are highlighted, both in the
path diagram (SEM) and in the GAM models. The year-to-year changes
between the indicators show the close relationship between the climate
indices and negative values for HCI (periods of warm conditions)
compared to existing positive values for SOI (El Niño phase). However,
HCI only manifests itself as a decadal trend in the prevailing
environmental conditions, while SOI represents the interannual variation
ENSO (El Niño-Southern Oscillation) (Blanco-Garcı'a, 2004).

The climate variables analyzed here should be considered as indicating
large-scale changes at the atmospheric level and in the ocean's climate
through ocean-atmosphere interaction. The influence at the habitat level
of the nylon shrimp on the continental slope is unknown. However,
climatic influence could likely influence the alteration in the
circulation patterns of the Peru-Chile Subsurface Current and the
distribution of physical properties associated with the mixing of
present water mass, i.e., Subsurface Equatorial and Antarctic
Intermediate (Nelson S. and Neshyba, 1979). These water mass intensify
or weaken during either El Niño or La Niña events, and influence the
subsurface dynamics where the nylon shrimp live (Hormazabal et al.,
2013).

The significance of these indicators is conditional and increases when
the incidence of common hake on spawning biomass in the year of
formation, and the recruitment rate is incorporated. Nevertheless, from
GLM models, we conclude that common hake would explain more than 90\% of
the variations in the recruitment rate.Recruitment sensitivity analysis
showed that common hake presented more considerable significance, which
expressed itself through a significant negative correlation in the
periods 1973-1975, 1977-1986, and from 2005 to 2011. These results are
consistent with (C Canales et al., 2016) in that the increase (decrease)
in shrimp biomass could be modulated in part by the depletion (increase)
of common hake. The biomass of common hake showed an increasing trend in
the first years of the series, with significant increases in 1972-1977,
1980-1984, 1991-1996, and 1998-2000 (Tascheri et al., 2017). These
periods coincide with the years of the highest association between hake
and the recruitment of nylon shrimp. Opposing trends between the
biomasses of common hake and spawning biomasses of nylon shrimp are
consistent with findings by Worm and Myers (2003), who show that the
abundance of northern shrimp (\emph{Pandalus Borealis}) was negatively
related to Atlantic cod (\emph{Gadus morhua}) populations. These authors
found that cod correlated with temperature, and they found that changes
in predator populations can have substantial effects on prey populations
in the ocean food networks and that the intensity of these interactions
can be sensitive to changes in mean ocean temperature (Worm and Myers,
2003). Björnsson et al. (2011) found that seasonal migratory activity of
immature \emph{G. morhua} was affected both by the location of the local
shrimp stock, and the seasonal and spatial differences in temperature.

We found that the incidence of shrimp in the diet of hake is generally
low (Arancibia et al., 1998; Cubillos et al., 2003), which could be
related to deeper distribution of naylon shrimp. Nevertheless, size
selectivity coefficients indicated that common hake from 2 to 4 years of
age preys preferentially on 1 to 3-year-old nylon shrimp (shrimp
juveniles, recruits, and spawning individuals, as reported by Canales et
al. (1999)). These results are consistent with those found by Cubillos
(2007), who conclude through the same analysis that hakes over 4 years
old prey on red squat lobster (\emph{Pleuroncodes monodon}) and yellow
squat lobster (\emph{Cervimunida johni}) over 2 years old. It is
important to note that after 2003, the biomass of common hake decreased
significantly and there was also a juvenilization of the age structure,
supported by specimens from 2 to 5 years of age, in addition to a
decrease in the size at maturity (Lillo et al., 2015). These demographic
changes in the predator could relax probable predation effects favoring
the recruitment of nylon shrimp.

With the above, studies of the fauna accompanying the direct assessment
cruises of common hake, point to nylon shrimp as one of the main species
within the faunal group of the demersal assembly (Acuña et al., 2019;
Lillo et al., 2015), the same occurs in the direct assessment cruises of
shrimp, where common hake and prawns, make up the main species
associated with the captures, with the former showing the most
significant recurrence in the captures throughout the entire sampling
area (Acuña et al., 2012, 2002, 2019; Arana et al., 2006).

The aforementioned suggests that the interaction in the biological
components would have a more significant impact on the biological
processes of the component species than the prevailing environmental
conditions.

\hypertarget{credit-authors}{%
\section{Credit authors}\label{credit-authors}}

\textbf{Camila Sagua}: Conceptualization, Investigation, Data curation,
Writing - Original draft preparation.

\textbf{Luis A. Cubillos}: Project Management, Formal analysis,
Methodology, Writing - Reviewing and Editing.

\textbf{Cristian M. Canales}: Data curation,Formal analysis, Writing -
Reviewing and Editing.

\textbf{Ruben Alarcón}: Formal analysis, Writing-Reviewing and Editing.

\hypertarget{acknowledgements}{%
\section{Acknowledgements}\label{acknowledgements}}

CS would like to thank to the Dirección de Postgrado of the Universidad
de Concepción for a scholarship during postgraduate studies, and LC
would like to thank to the COPAS Sur-Austral CONICYT PIA APOYO CCTE
AFB170006 for partial fundings. All code used to generate this paper, as
well as prior versions of this manuscript, are available at:
\href{https://github.com/LuisACubillos/shrimp-climate-predation}{github.com/LuisACubillos/shrimp-climate-predation}.

\FloatBarrier

\hypertarget{bibliography}{%
\section*{Bibliography}\label{bibliography}}
\addcontentsline{toc}{section}{Bibliography}

\hypertarget{refs}{}
\leavevmode\hypertarget{ref-Acunaetal2012}{}%
Acuña, E., Alarcón, R., Cortés, A., Arancibia, H., Cubillos, L., Cid,
L., 2012. Evaluación directa de camarón nailon entre la ii y viii
regiones, año 2011 (No. FIP 2011-02). Fondo de Investigación Pesquera y
Acuicultura.

\leavevmode\hypertarget{ref-Acunaetal:2002}{}%
Acuña, E., Arancibia, H., Cid, L., Alarcón, R., Cubillos, L., Sepúlveda,
A., Bodini, A., Bennett, X., Cabrera, M., Villarroal, J., León, R.,
Wiff, R., Grau, R., Andrade, M., Casas, L., Rivera, D., Poblete, L.,
Vásquez, G., 2002. Evaluación directa de camarón nailon entre la ii y
viii regiones, año 2001 (Final Report No. FIP 2001-05). Fondo de
Investigación Pesquera y Acuicultura.

\leavevmode\hypertarget{ref-Acunaetal2019}{}%
Acuña, E., Haye, P., Segovia, N., Arancibia, H., Sagua, C., Zuñiga, A.,
Alarcón, R., Cortés, A., Cid, L., Petigas, P., 2019. Evaluación directa
de camarón nailon entre la ii y viii regiones, año 2018 (Final Report).
Instituto de Fomento Pesquero.

\leavevmode\hypertarget{ref-Akaike1973}{}%
Akaike, H., 1973. Information theory and an extension of the maximum
likelihood principle, in: Petrov, B., Csaki, F. (Eds.), Second
International Symposium on Information Theory. Akademiai Kiado, pp.
267--281.

\leavevmode\hypertarget{ref-ALHEIT2004201}{}%
Alheit, J., Niquen, M., 2004. Regime shifts in the humboldt current
ecosystem. Progress in Oceanography 60, 201--222.
doi:\href{https://doi.org/https://doi.org/10.1016/j.pocean.2004.02.006}{https://doi.org/10.1016/j.pocean.2004.02.006}

\leavevmode\hypertarget{ref-Arana2012}{}%
Arana, P.M., 2012. Recursos pesqueros del mar de chile. Escuela de
Ciencias del Mar, Pontificia Universidad Católica de Valparaı'so.

\leavevmode\hypertarget{ref-Aranaetal:2006}{}%
Arana, P.M., Ahumada, M., Guerrero, A., Melo, T., Queirolo, D.,
Barbieri, M., Bahamonde, N., Canales, C., Quiroz, J., 2006. Evaluación
directa de camarón nailon y gamba entre la ii y viii regiones, año 2005
(Final Report No. FIP 2005-08). Fondo de Investigación Pesquera y
Acuicultura.

\leavevmode\hypertarget{ref-AranaWilliams1970}{}%
Arana, P., Williams, S., 1970. Contribución al conocimiento del régimen
alimentario de la merluza (merluccius gayi). Investigaciones Marinas 1,
139--154.

\leavevmode\hypertarget{ref-Arancibiaetal1998}{}%
Arancibia, H., Catrilao, M., Farı'as, B., 1998. Evaluación de la demanda
de alimento en merluza común y análisis de su impacto en pre-reclutas.
(No. FIP 95-17). Fondo de Investigación Pesquera y Acuicultura.

\leavevmode\hypertarget{ref-ArancibiaNeira2008}{}%
Arancibia, H., Neira, S., 2008. Overview of the chilean hake (merluccius
gayi) stock, a biomass forecast, and the jumbo squid (dosidicus gigas)
predator-prey relationship off central chile (33ºS-39ºS). CalCOFI Rep.
49, 104--115.

\leavevmode\hypertarget{ref-Ayon2011}{}%
Ayón, P., Swartzman, G., Espinoza, P., Bertrand, A., 2011. Long-term
changes in zooplankton size distribution in the peruvian humboldt
current system: Conditions favouring sardine or anchovy. Marine Ecology
Progress Series 422, 211--222.

\leavevmode\hypertarget{ref-Bahamonde:1970}{}%
Bahamonde, N., Henrı'quez, G., 1970. Sinopsis de datos biológicos sobre
el camarón nailon heterocarpus reedi, bahamonde 1955., in: Mistakidis,
M.N. (Ed.), Proceedings of the World Scientific Conference on the
Biology and Culture of Shrimps and Prawns, FAO Fishing Report. FAO; FAO,
pp. 1607--1627.

\leavevmode\hypertarget{ref-B:2017aa}{}%
Björnsson, B., Burgos, J., Sólmundsson, J., Ragnarsson, S., Jónsdóttir,
I., Skúladóttir, U., 2017. Effects of cod and haddock abundance on the
distribution and abundance of northern shrimp. Marine Ecology Progress
Series 572, 209--221.

\leavevmode\hypertarget{ref-Bjornsson2011}{}%
Björnsson, B., Reynisson, P., Solmundsson, J., Valdimarsson, H., 2011.
Seasonal changes in migratory and predatory activity of two species of
gadoid preying on inshore northern shrimp pandalus borealis. Journal of
Fish Biology 78, 1110--1131.
doi:\href{https://doi.org/10.1111/j.1095-8649.2011.02923.x}{10.1111/j.1095-8649.2011.02923.x}

\leavevmode\hypertarget{ref-Blanco2004}{}%
Blanco-Garcı'a, J.L., 2004. Inter-annual to inter-decadal variability of
upwelling and anchovy population off northern chile (PhD thesis).
Ocean/Earth/Atmos Sciences, Old Dominion University.

\leavevmode\hypertarget{ref-Buckland1997}{}%
Buckland, S.T., Burnham, K.P., Augustin, N.H., 1997. Model selection: An
integral part of inference. Biometrics 53, 603--618.

\leavevmode\hypertarget{ref-CAHUIN201388}{}%
Cahuin, S.M., Cubillos, L.A., Escribano, R., Blanco{]}, J. {[}Luis,
Ñiquen, M., Serra, R., 2013. Sensitivity of recruitment rates anchovy
(engraulis ringens) to environmental changes in southern peru---northern
chile. Environmental Development 7, 88--101.
doi:\href{https://doi.org/https://doi.org/10.1016/j.envdev.2013.03.004}{https://doi.org/10.1016/j.envdev.2013.03.004}

\leavevmode\hypertarget{ref-Canales2016aa}{}%
Canales, C., Company, J., Arana, P., 2016. Population structure of nylon
shrimp Heterocarpus reedi (crustacea: Caridea) and its relationship with
environmental variables off chile. Latin American Journal of Aquatic
Research 44, 144--154.
doi:\href{https://doi.org/10.3856/vol44-issue1-fulltext-15}{10.3856/vol44-issue1-fulltext-15}

\leavevmode\hypertarget{ref-CANALES2016360}{}%
Canales, C., Company, J.B., Arana, P.M., 2016. Using a length-based
stock assessment model to evaluate population structure hypotheses of
nylon shrimp heterocarpus reedi (decapoda, caridea) exploited off
central chile. Fisheries Research 183, 360--370.
doi:\href{https://doi.org/https://doi.org/10.1016/j.fishres.2016.06.020}{https://doi.org/10.1016/j.fishres.2016.06.020}

\leavevmode\hypertarget{ref-CANALES20161}{}%
Canales, C.M., Company, J., Arana, P., 2016. Spatio-temporal modelling
of the maturity, sex ratio, and physical condition of nylon shrimp
heterocarpus reedi (decapoda, caridea), off central chile. Fisheries
Research 179, 1--9.
doi:\href{https://doi.org/https://doi.org/10.1016/j.fishres.2016.02.001}{https://doi.org/10.1016/j.fishres.2016.02.001}

\leavevmode\hypertarget{ref-Canalesetal:1999}{}%
Canales, C., Montenegro, C., Peñailillo, T., Pool, H., 1999. Evaluación
indirecta del stock de camarón nailon en el litoral de la ii a viii
regiones (Final Report No. FIP 97-24). Fondo de Investigación Pesquera y
Acuicultura.

\leavevmode\hypertarget{ref-Chavez217}{}%
Chavez, F.P., Ryan, J., Lluch-Cota, S.E., C., M. Ñiquen, 2003. From
anchovies to sardines and back: Multidecadal change in the pacific
ocean. Science 299, 217--221.
doi:\href{https://doi.org/10.1126/science.1075880}{10.1126/science.1075880}

\leavevmode\hypertarget{ref-CUBILLOS2007}{}%
Cubillos, C.A.A., Luis A AND Alarcoń, 2007. Selectividad por tamaño de
las presas en merluza comuń (Merluccius gayi gayi), zona centro-sur de
Chile (1992-1997). Investigaciones marinas 35, 55--69.

\leavevmode\hypertarget{ref-Cubillos2007a}{}%
Cubillos, L.A., Serra, R., Fréon, P., 2007. Synchronous pattern of
fluctuation in three anchovy fisheries in the humboldt current system.
Aquat. Living Resour. 20, 69--75.
doi:\href{https://doi.org/10.1051/alr:2007017}{10.1051/alr:2007017}

\leavevmode\hypertarget{ref-Cubillosetal:2003mgayi}{}%
Cubillos, L., Rebolledo, H., Hernández, A., 2003. Prey composition and
estimation of q/b for the chilean hake, merluccius gayi
(gadiformes-merluccidae), in the central-south area off chile
(34º--40ºS). Archive of Fishery and Marine Research 50, 271--286.

\leavevmode\hypertarget{ref-Cury2000}{}%
Cury, P., Bakun, A., Crawford, R.J.M., Jarre, A., Quiñones, R.A.,
Shannon, L.J., Verheye, H.M., 2000. Small pelagics in upwelling systems:
patterns of interaction and structural changes in ``wasp-waist''
ecosystems. ICES Journal of Marine Science 57, 603--618.
doi:\href{https://doi.org/10.1006/jmsc.2000.0712}{10.1006/jmsc.2000.0712}

\leavevmode\hypertarget{ref-DRINKWATER2010374}{}%
Drinkwater, K.F., Beaugrand, G., Kaeriyama, M., Kim, S., Ottersen, G.,
Perry, R.I., Pörtner, H.-O., Polovina, J.J., Takasuka, A., 2010. On the
processes linking climate to ecosystem changes. Journal of Marine
Systems 79, 374--388.
doi:\href{https://doi.org/https://doi.org/10.1016/j.jmarsys.2008.12.014}{https://doi.org/10.1016/j.jmarsys.2008.12.014}

\leavevmode\hypertarget{ref-Gomez:2012}{}%
Gómez, F., Montecinos, A., Hormazabal, S., Cubillos, L.A.,
Correa-Ramirez, M., Chavez, F.P., 2012. Impact of spring upwelling
variability off southern-central chile on common sardine (strangomera
bentincki) recruitment. Fisheries Oceanography 21, 405--414.
doi:\href{https://doi.org/10.1111/j.1365-2419.2012.00632.x}{10.1111/j.1365-2419.2012.00632.x}

\leavevmode\hypertarget{ref-Graceetal:2010}{}%
Grace, J.B., Anderson, T.M., Olff, H., Scheiner, S.M., 2010. On the
specification of structural equation models for ecological systems.
Ecological Monographs 80, 67--87.
doi:\href{https://doi.org/10.1890/09-0464.1}{10.1890/09-0464.1}

\leavevmode\hypertarget{ref-GraceBollen2005}{}%
Grace, J.B., Bollen, K.A., 2005. Interpreting the results from multiple
regression and structural equation models. Bulletin of the Ecological
Society of America 86, 283--295.

\leavevmode\hypertarget{ref-Hidalgo:2011aa}{}%
Hidalgo, M., Rouyer, T., Molinero, J., Massutı', E., Moranta, J.,
Guijarro, B., Stenseth, N.C., 2011. Synergistic effects of
fishing-induced demographic changes and climate variation on fish
population dynamics. Marine Ecology Progress Series 426, 1--12.

\leavevmode\hypertarget{ref-Hormazabal:2013}{}%
Hormazabal, S., Combes, V., Morales, C.E., Correa-Ramirez, M.A., Di
Lorenzo, E., Nuñez, S., 2013. Intrathermocline eddies in the coastal
transition zone off central chile (31--41\(\,^{\circ}\)S). Journal of
Geophysical Research: Oceans 118, 4811--4821.
doi:\href{https://doi.org/10.1002/jgrc.20337}{10.1002/jgrc.20337}

\leavevmode\hypertarget{ref-Hsieh:2006aa}{}%
Hsieh, C.-h., Reiss, C.S., Hunter, J.R., Beddington, J.R., May, R.M.,
Sugihara, G., 2006. Fishing elevates variability in the abundance of
exploited species. Nature 443, 859--862.
doi:\href{https://doi.org/10.1038/nature05232}{10.1038/nature05232}

\leavevmode\hypertarget{ref-Lehodey2006}{}%
Lehodey, P., Alheit, J., Barange, M., Baumgartner, T., Beaugrand, G.,
Drinkwater, K., Fromentin, J.-M., Hare, S.R., Ottersen, G., Perry, R.I.,
Roy, C., Lingen, C.D. van der, Werner, F., 2006. Climate Variability,
Fish, and Fisheries. Journal of Climate 19, 5009--5030.
doi:\href{https://doi.org/10.1175/JCLI3898.1}{10.1175/JCLI3898.1}

\leavevmode\hypertarget{ref-Lilloetal2015}{}%
Lillo, S., Legua, J., Olivares, J., Saavedra, J., Molina, E., Rojas, M.,
Angulo, J., Valenzuela, V., Nuñez, S., Vásquez, S., 2015. Evaluación
directa de merluza común, 2014 (Final Report). Instituto de Fomento
Pesquero.

\leavevmode\hypertarget{ref-McCullaghNelder1989}{}%
McCullagh, P., Nelder, J., 1989. Generalized linear models, Monographs
on statistics and applied probability 37. Chapman; Hall.

\leavevmode\hypertarget{ref-MONTENEGRO201648}{}%
Montenegro, C., Branco, M., 2016. Bayesian state-space approach to
biomass dynamic models with skewed and heavy-tailed error distributions.
Fisheries Research 181, 48--62.
doi:\href{https://doi.org/https://doi.org/10.1016/j.fishres.2016.03.021}{https://doi.org/10.1016/j.fishres.2016.03.021}

\leavevmode\hypertarget{ref-S19791387}{}%
Nelson S., S., Neshyba, S., 1979. On the southernmost extension of the
peru-chile undercurrent. Deep Sea Research Part A. Oceanographic
Research Papers 26, 1387--1393.
doi:\href{https://doi.org/https://doi.org/10.1016/0198-0149(79)90006-2}{https://doi.org/10.1016/0198-0149(79)90006-2}

\leavevmode\hypertarget{ref-OVERLAND2010305}{}%
Overland, J.E., Alheit, J., Bakun, A., Hurrell, J.W., Mackas, D.L.,
Miller, A.J., 2010. Climate controls on marine ecosystems and fish
populations. Journal of Marine Systems 79, 305--315.
doi:\href{https://doi.org/https://doi.org/10.1016/j.jmarsys.2008.12.009}{https://doi.org/10.1016/j.jmarsys.2008.12.009}

\leavevmode\hypertarget{ref-PARADA2013}{}%
Parada, B.A.H., Carolina AND Yannicelli, 2013. Variabilidad ambiental y
recursos pesqueros en el PacÃfico suroriental: estado de la
investigaciÃy desafÃos para el manejo pesquero. Latin american journal
of aquatic research 41, 1--28.

\leavevmode\hypertarget{ref-PERRY2010427}{}%
Perry, R.I., Cury, P., Brander, K., Jennings, S., Möllmann, C., Planque,
B., 2010. Sensitivity of marine systems to climate and fishing:
Concepts, issues and management responses. Journal of Marine Systems 79,
427--435.
doi:\href{https://doi.org/https://doi.org/10.1016/j.jmarsys.2008.12.017}{https://doi.org/10.1016/j.jmarsys.2008.12.017}

\leavevmode\hypertarget{ref-PLANQUE2010403}{}%
Planque, B., Fromentin, J.-M., Cury, P., Drinkwater, K.F., Jennings, S.,
Perry, R.I., Kifani, S., 2010. How does fishing alter marine populations
and ecosystems sensitivity to climate? Journal of Marine Systems 79,
403--417.
doi:\href{https://doi.org/https://doi.org/10.1016/j.jmarsys.2008.12.018}{https://doi.org/10.1016/j.jmarsys.2008.12.018}

\leavevmode\hypertarget{ref-Silva2012}{}%
Silva, N., 2012. Recursos pesqueros del mar de chile, in: Arana, P.M.
(Ed.),. Escuela de Ciencias del Mar, Pontificia Universidad Católica de
Valparaı'so, pp. 39--52.

\leavevmode\hypertarget{ref-Tascheri2017}{}%
Tascheri, R., Gálvez, P., Sateler, J., 2017. Estatus y posibilidades de
explotación biológicamente sustentables de los principales recursos
pesqueros nacionales al año 2016: Merluza común, 2016. Instituto de
Fomento Pesquero, Valparaı'so, Chile.

\leavevmode\hypertarget{ref-Ursin1973}{}%
Ursin, E., 1973. On the prey size preference of cod and dab. Medd. Dan.
Fisk.-Havunders 7, 85--98.

\leavevmode\hypertarget{ref-vanderLingen2009}{}%
van der Lingen, C., Bertrand, A., Bode, A., Brodeur, R., Cubillos, L.,
Espinoza, P., Friedland, K., Garrido, S., Irigoien, X., Miller, T.,
Möllmann, C., Rodriguez-Sanchez, R., Tanaka, H., Temming, A., 2009.
Trophic dynamics, in: Checkley, D., Alheit, J., Oozeki, Y., Roy, C.
(Eds.),. Cambridge University Press, pp. 112--157.

\leavevmode\hypertarget{ref-VenablesRipley:2002}{}%
Venables, W., Ripley, B., 2002. Modern applied statistics with s, 4th
edn. ed. Springer, New York.

\leavevmode\hypertarget{ref-Worm:2003}{}%
Worm, B., Myers, R.A., 2003. Meta-analysis of cod-shrimp interactions
reveals top-down control in oceanic food webs. Ecology 84, 162--173.

\leavevmode\hypertarget{ref-XU2019103229}{}%
Xu, Y., Fu, C., Peña, A., Hourston, R., Thomson, R., Robinson, C.,
Cleary, J., Daniel, K., Thompson, M., 2019. Variability of pacific
herring (clupea pallasii) spawn abundance under climate change off the
west coast of canada over the past six decades. Journal of Marine
Systems 200, 103229.
doi:\href{https://doi.org/https://doi.org/10.1016/j.jmarsys.2019.103229}{https://doi.org/10.1016/j.jmarsys.2019.103229}

\leavevmode\hypertarget{ref-Yanez2008}{}%
Yañez, E., Hormazábal, S., Silva, C., Montecinos, A., Barbieri, M.A.,
Valdenegro, A., Órdenes, A., Gómez, F., 2008. Coupling between the
environment and the pelagic resources exploited off northern chile:
Ecosystem indicators and a conceptual model. Latin American Journal of
Aquatic Research 36, 159--181.
doi:\href{https://doi.org/10.3856/vol36-issue2-fulltext-3}{10.3856/vol36-issue2-fulltext-3}

\end{document}
